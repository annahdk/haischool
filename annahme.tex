\documentclass[11pt,a4paper]{article}
\usepackage[utf8]{inputenc}
\usepackage[german]{babel}
\usepackage{amsmath}
\usepackage[table]  {xcolor}
\usepackage{graphicx}
\usepackage { tabularx }
\usepackage{amsfonts}
\usepackage[paper=a4paper,left=20mm,right=20mm,top=20mm,bottom=20mm]{geometry}
\usepackage{amssymb}
\usepackage{blindtext}   % Für Beispieltexte
\usepackage{enumitem}


\author{Tim Ritter}


\begin{document}

\begin{table}[h!]
\caption{Annahmen}\bigskip
\centering
\begin{tabular}{l|l}
\textbf{Problem} & \textbf{Erklärung} \\
Unabhängig- & könnte abhängig sein weil none ungleich nicht teilgenommen \\
keit  & aber wir nehmen unabhängig an weil es unterschiedliche Personen sind\\
stetige &  vereinfachte Analyse der Daten, shapiro wilk test lehnt die normalverteilung nicht ab (für alle),\\
scores & qqplot und Histogramm deutet auf eine NV hin, Rangordnung (100 besser als 0 Punkte), \\
pro & interpretierbare Abstände, arithmetisches Mittel verwendbar wegen intervallskaliert   \\
stetige scores & kritischer p-Wert, qqplot hat Ausreisser, Wertebereich ist von 0-100; \\
 contra & sollte aber zwischen -$\inf$ bis $\inf$,  \\
Geschlecht nominal & weil kein Rang \\
Ethnie nominal & weil kein Rang \\
Bildung Eltern ordinal & weil Rang \\
Mittagsessen nominal & weil kein Rang \\
Vorkurs nominal & weil kein Rang \\
Leistungen kardinal & weil normalverteilt \\
statistische Methoden & nach präsi dazu


\end{tabular}
\end{table}

statistische Methoden: p-Wert ?, statistischer Test?, Fehler 1./2.Art?, Verteilungen ?, niveau adjustierung?
\end{document}